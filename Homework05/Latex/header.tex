\documentclass{article} % Default font size and left-justified equations
% \documentclass{book}

\usepackage[
  margin=0.7in,
  % includefoot,
  % footskip=30pt,
]{geometry}

\usepackage{natbib}
\usepackage{enumitem}
\usepackage{amsmath,amsfonts}
\usepackage{graphicx}
\usepackage{framed}
\usepackage{subcaption}
\usepackage[T1]{fontenc}%
\usepackage[utf8]{inputenc}%
\usepackage{mathrsfs}%
\usepackage{amssymb}%
\usepackage{amsthm}%
\usepackage{graphicx}
\usepackage{hyperref}
\usepackage{environ}
\usepackage{framed}
\usepackage{mdframed} 
\usepackage{wrapfig} 
\usepackage{booktabs}
\usepackage{tabularx} 
\usepackage{array}
\usepackage[font=small]{caption} 
\usepackage{xspace}  
\usepackage[osf,sc]{mathpazo}
\usepackage{pbox}
\usepackage{listings}
\usepackage{tikz}
\usepackage{bm}
\usepackage[strict]{changepage}
\usepackage{mleftright}
\usepackage{appendix}%
\usepackage[all]{xy} 
\setcounter{tocdepth}{3}
\setcounter{secnumdepth}{3}
\usepackage[ruled,vlined]{algorithm2e}
\usepackage{framed}
\usepackage{wrapfig}
\usepackage{pythonhighlight}
\usepackage{tikz}
\usetikzlibrary{arrows}
\usetikzlibrary{arrows.meta}
\usetikzlibrary{shapes.geometric}
\usetikzlibrary{positioning, arrows, automata, calc}
\usepackage{transparent}
\usepackage[many]{tcolorbox}
\usepackage{tikz}
\usetikzlibrary{shapes.geometric}
\usetikzlibrary{positioning, arrows, automata, calc}

\newtcolorbox[]{your_solution}[1][]{
    % breakable,
    enhanced,
    nobeforeafter,
    colback=white,
    title=Your Answer,
    sidebyside align=top,
    box align=top,
    #1
}
\newcommand{\mybox}[1]{
\noindent
\fbox{\parbox{0.955\textwidth}{%
\noindent\texttt{#1}
}
}
}

\newcommand{\filler}{ . . . . . }
\newcommand{\choice}{\hspace{0.5cm}$\square$}
\newcommand{\identity}{\mathbf{I}} 
\newcommand{\paran}[1]{\left( #1 \right)}

%% PLEASE USE THESE MACROS WHEN WRITING THE SOLUTIONS 
\newcommand{\solution}[1]{\textcolor{blue}{Answer: \em #1}
} % show
%\newcommand{\solution}[2]{#2} % hide

\newcommand{\extracredit}{{\color{purple}\textbf{Extra Credit:}}}

\newcommand{\additionalNotes}{
\noindent\textbf{How to hand in your written work:}   \\ 


\noindent\textbf{Collaboration:} Make certain that you understand the course collaboration policy, described on the course website. 
You may discuss the homework to understand the problems and the mathematics behind the various learning algorithms, but you are \textcolor{red}{\textbf{not allowed to share problem solutions with any other students. You must write the solutions \textbf{individually}}}. \\ 

}

\newcommand{\additionalNotesTeamUpTwo}{
\noindent\textbf{How to hand in your written work:} Via MyClasses as before.  \\ 


\noindent\textbf{Collaboration:} You can do this homework as a team of 2 people. Basically, you can write a single solution for both people and upload it as a single PDF and a single \texttt{.zip} file as a group. \\ 

}

\title{ \Large COSC490LLModels }
\date{ \\ 
\normalsize Name:   \_\_\_Kyle Tranfaglia\_\_\_\   \\ 
Collaborators, if any: \_\_\_\_NA\_\_\_\_\_ \\ 
Sources used for your homework, if any: \_\_\_ Claude AI for code debugging and math assistance \_\_\_\ 
}

\newcommand{\todo}{\textcolor{blue}{\textbf{TODOs}}}

